\chapter{Wstęp}
\label{chapter:1}

\section{Zawartość pracy}

\hyperref[chapter:1]{Pierwszy rozdział} zawiera wprowadzenie wyjaśniające jaki problem rozwiązuje aplikacja \emph{Poker Master Tool}.
\hyperref[chapter:2]{Drugi rozdział} objaśnia zasady gry Poker Texas Hold’em, wyjaśnia działanie kalkulatorów szans oraz zawiera porównanie z innymi istniejącymi rozwiązaniami.
\hyperref[chapter:3]{Trzeci rozdział} opisuje algorytm wykorzystany w silniku aplikacji.
W \hyperref[chapter:4]{czwartym rozdziale} omówiono implementację rozwiązania.
\hyperref[chapter:5]{Piąty rozdział}  to instrukcja użytkownika z demonstracyjnymi przypadkami użycia.
W \hyperref[chapter:6]{szóstym rozdziale} zostało zawarte podsumowanie.

\section{Opis problemu i cel powstania aplikacji}

Poker \cite{wiki-poker} to jedna z najpopularniejszych gier karcianych na świecie. Codziennie grają w niego miliony graczy online oraz na żywo. Kluczowymi umiejętnościami jest dobór odpowiedniej strategii, znajomość elementów rachunku prawdopodobieństwa oraz aspekty psychologiczne. Aby udoskonalać swoją strategię, profesjonalni pokerzyści korzystają z szerokiej gamy narzędzi. Jedną z kategorii takich aplikacji są kalkulatory szans (ang. equity calculator), które służą do obliczenia szans wygranej dla poszczególnych graczy przy zadanych kartach graczy oraz kartach wspólnych. Dzięki temu gracze mogą ocenić siłę swojej ręki co ułatwia im podejmowanie decyzji.

Niestety, większość kalkulatorów obecnych na rynku jest nieprzyjazna dla amatorów pokera --- zawierają zbyt dużo skomplikowanych opcji, a ich interfejs jest nieczytelny, przez co próg wejścia jest wysoki, jeżeli gracz zechciałby zacząć ich używać. Większość kalkulatorów jest dostępna tylko jako aplikacja desktopowa, a inne w formie aplikacji webowej (sieciowej) są nieprzystosowane do szerokości ekranów urządzeń mobilnych. 

\emph{Poker Master Tool} stara się rozwiązać ten problem --- jest to w pełni responsywna aplikacja webowa, z interfejsem dostosowanym do współczesnych standardów, oferująca kalkulację szans na wygraną i szans uzyskania poszczególnych układów. Dzięki prostocie obsługi kalkulatora, mniej doświadczeni gracze mają szansę z zaznajomieniem się z teorią pokera. 

Aplikacja dostępna jest pod adresem \href{https://pokermastertool.bartoszputek.pl/}{https://pokermastertool.bartoszputek.pl/}, a kod źródłowy na repozytorium \href{https://github.com/bartoszputek/poker-master-tool/}{https://github.com/bartoszputek/poker-master-tool/}.
